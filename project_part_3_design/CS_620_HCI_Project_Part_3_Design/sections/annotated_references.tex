\section{Annotated References and Resources}
\label{sect:annotated_references}

Listed below are annotated references which helped produce this document covering PyBank's design. Two of the references, \emph{The Design of Everyday Things}~\cite{THE_DESIGN_OF_EVERYDAY_THINGS:0} and \emph{Designing the User Interface}~\cite{DESIGNING_THE_USER_INTERFACE:6} are utilized by CS 420/620 taught by Dr. Sergiu Dascalu. The other five annotated references helped produce our diagrammatic solutions or have aided with the development of PyBank. All annotated references contain hyperlinks to the \textbf{References}~section which contains the bibliographic information for each reference or resource found at the end of this document.

\subsection{The Design of Everyday Things}
\label{sect:doet}

\emph{The Design of Everyday Things}~\cite{THE_DESIGN_OF_EVERYDAY_THINGS:0} is a book that details good and bad designs in everyday things and presents fundamental properties of good design that are applicable across domains. For PyBank, we've utilized two concepts heavily during our design process. The first concept, \emph{signifiers}, are described by the author Don Norman as ``[...] any mark or sound, any perceivable indicator that communicates appropriate behavior to a person~\cite{THE_DESIGN_OF_EVERYDAY_THINGS:0}." Our design incorporates signifiers through the use of a variety of PyQt5 widgets such as buttons, labels, text boxes, etc. All of these widgets serve to communicate to a PyBank customer the appropriate behavior for an action within the application. We also use another concept presented in the book, \emph{feedback}, to help provide a user with immediate communication related to the results of an action. PyBank accomplishes this feat through the use of error and information dialog boxes.

\subsection{Designing the User Interface}
\label{sect:dtui}

\emph{Designing the User Interface}~\cite{DESIGNING_THE_USER_INTERFACE:6} provides strategies for effective human-computer interaction. This book covers usability, guidelines, principles, theories, the design process, interaction styles, and design issues. One important concept covered in Chapter 4, Section 4.5.5 is prototyping~\cite{DESIGNING_THE_USER_INTERFACE:6}. The book presents three types, low-fidelity, medium-fidelity, and high-fidelity prototypes. Our static interface designs found in Section~\ref{sect:static_interface_design} and our alternative designs found in Section~\ref{sect:alternative_designs} utilized Balsamiq Mockups 3~\cite{BALSAMIQ_MOCKUPS_3:1} which helped us create our medium-fidelity prototypes.

\subsection{Balsamiq Mockups 3}
\label{sect:balsamiq_mockups_3}

\emph{Balsamiq Mockups 3}~\cite{BALSAMIQ_MOCKUPS_3:1} is a software application that provides rapid prototyping through wireframing. One feature of this application are UI Controls such as buttons, text input, drop down menus, radio buttons, menu bars, icons, and many more controls. Another feature of \emph{Balsamiq Mockups 3}~\cite{BALSAMIQ_MOCKUPS_3:1} is the ability to share wireframes with other designers and collaborate across platforms and devices using the technologies like Google Drive, the cloud version of the application, among other things.

\subsection{Lucid Chart}
\label{sect:lucid_chart}

\emph{Lucid Chart}~\cite{LUCID_CHART:2} is a cloud-based diagramming and visualization software solution that is extremely popular. It is endorsed by 99\% of fortune 500 companies and it provides over 500 templates for speeding up the diagramming process. \emph{Lucid Chart}~\cite{LUCID_CHART:2} supports multi-page diagram solutions and allows many designers to collaborate together on the same project. An additional benefit of \emph{Lucid Chart}~\cite{LUCID_CHART:2} is that it supports templates for technologists and business professionals making it a solution for a variety of practitioners.

\subsection{Qt}
\label{sect:qt}

\emph{Qt}~\cite{QT:3} is the official source for documentation for the Qt API. In our case, our work relates specifically to the Python API for Qt. Qt more broadly is a cross-platform software solution for embedded systems and desktop systems. The documentation provided by \emph{Qt}~\cite{QT:3} includes: examples, tutorials, quick-start guides, development tools, and the Qt Creator IDE which supports ``[...] a powerful cross-platform integrated development environment, including UI designer tools and on-device debugging~\cite{QT:3}."

\subsection{ZetCode}
\label{sect:zetcode}

\emph{ZetCode}~\cite{ZETCODE:4} ``[...] brings tutorials for programmers in various areas. The main are Graphical User Interfaces, databases, and programming languages. The website's mission is to provide competent, quick and easy to understand tutorials for modern-day technologies~\cite{ZETCODE:4}." This website is maintained by its author, Jan Bodnar. It has provided useful tutorials for the development of PyBank.

\subsection{Material Design}
\label{sect:material_design}

\emph{Material}~\cite{MATERIAL:7} ``[...] Material is a design system – backed by open-source code – that helps teams build high-quality digital experiences~\cite{MATERIAL:7}." Material provides four main selections from the homepage which are the following: Design, Components, Develop, and Resources. All of these selections have many sub-categories. Material provides a vast set of design tips based on its philosophy for creating user interfaces. Everything from layout to navigation to typography, etc. is covered by the website. Material also provides icons, Google fonts, and many other resources.

%EOF