\section{Requirements Discovery}
\label{sect:requirements_discovery}

Three users were interviewed regarding Pybank to determine what potential users would like to see from this application. The stakeholders who took part in this process were: Chase Carthen, Joseph Masset, and Vinh Le.


%----- START QUESTION 1 ----%
    \bigskip\noindent\textit{"What do you like about any banking or budgeting applications you have seen or utilized yourself?"}
    
    \bigskip\noindent In response to this question, two users both mentioned that they enjoy being able to view their accounts and associated transactions. Chase elaborated that he specifically likes when banks offer options to auto group the transactions as well. On the other hand, Vinh noted the convenience of mobile banking instead, stating that he enjoys being able to scan checks and payments digitally without having to travel to a local bank. 
%----- END QUESTION 1 ------%

%----- START QUESTION 2 ----%
    \bigskip\noindent\textit{"What qualities do you not like about current banking or budgeting application and web clients that you currently use or have seen?"}

    \bigskip\noindent All three users focused on different qualities of applications that they currently utilize. Joseph commented that he dislikes how long it take for data to pool from each location, resulting in inaccuracy of account summaries and transaction details while waiting for the summary to reflect the actual balance. Chase mentioned that he dislikes how functionality for many applications and websites directly employed by banks do not always reflect one another. For this reason, one version may not be able to support use by specific users. Lastly, Vinh responded that the lack of notifications from current banking apps and methods of notification delivery when an alert system is in place can be frustrating. He suggested that notifications for upcoming payments that display directly on a phone or computer would be very useful.
%----- END QUESTION 2 ------%

%----- START QUESTION 3 ----%
    \bigskip\noindent\textit{"If you had the option to have your banking and transaction information displayed visually in a graph, what would you like to see displayed?"}
    
    \bigskip\noindent Two main suggestions were made by the users that were interviewed regarding this application. First, users mentioned that they would find a pie chart or bar graph showing a monthly breakdown of expenditures and budgeting based on transaction purpose to be helpful. Secondly, a line graph that shows income versus expenses per month was recommended in order to visually show successful and less successful saving and profits.
%----- END QUESTION 3 ------%

%----- START QUESTION 4 ----%
    \bigskip\noindent\textit{"What is your favorite type of graph to receive information visually and why?"}
    
    \bigskip\noindent A general consensus was reach by all users that they had a preference for bar graphs and line graphs when looking at financial data. It was stated that, due to clearly marked divisions in categories, it is easy to compare spending of different types when viewing a bar graph. On the other hand, line graphs are better for continuous data and users find their analysis of trends easy to understand.
%----- END QUESTION 4 ------%

%----- START QUESTION 5 ----%
    \bigskip \noindent\textit{"What is the main operating system that you prefer to utilize?"}
    
    \bigskip \noindent All three users interviewed utilize Windows and Linux as their primary operating systems and favor Windows for regular use. For this reason, they expressed interest in the application being compatible with both Windows and Linux; however, in the case that cross-platform functionality cannot be achieved, the users would prefer a Windows client.
%----- END QUESTION 5 ------%

%----- START QUESTION 6 ----%
    \bigskip\noindent\textit{"When utilizing a banking or budgeting application, on average what is the primary function you utilize that it provides? (Ex: Account summaries, transaction details, etc)"}
    
    \bigskip\noindent All three users utilize banking applications with the same primary purpose, to see current account summaries and balances. While individual transaction details and transferring money between accounts were also mentioned in response, they were labelled as secondary to an overall summary of account statuses.
%----- END QUESTION 6 ------%

%----- START QUESTION 7 ----%
    \bigskip\noindent\textit{"Which features do you rarely utilize in banking or budgeting applications?"}

    \bigskip\noindent All three users have different features that they rarely utilize in banking applications, providing different options for deciding priority among functions. Joseph commented that he rarely uses options to follow or graph a budget due to lack of granular control that results in incorrect reporting. Chase instead noted that he does not utilize functions to create spending caps as they are not useful. Lastly, Vinh stated that he does not check his monthly statement on banking applications because it downloads a PDF containing sensitive account information locally to a device. This option is not secure and allows easy access to user information.
%----- END QUESTION 7 ------%    

%----- START QUESTION 8 ----%
    \bigskip\noindent\textit{"When adding a transaction, what information do you find necessary to provide to a banking or budgeting application?"}

    \bigskip\noindent Users responded stating that the most important information to provide are transaction amount, vendor name, account pulled from, and transaction type, such as rent, groceries, or entertainment. Joseph also recommended the addition of an optional notes section that can be regularly edited. This allows the user to include additional details about the transaction that would not typically be provided to better organize why the expenditures are necessary or to create more granular types within the broader spectrum to which the transactions fit. While he has not seen this in practice, he feels it would be a helpful addition to current transaction recording forms.
%----- END QUESTION 8 ------%

%----- START QUESTION 9 ----%
    \bigskip\noindent\textit{"What information are you often asked for when filling forms do you find superfluous?"}
    
    \bigskip\noindent A variety of issues were mentioned surrounding various form fields utilized by banks. In particular, Joseph and Vinh were concerned with personal identifying information that is often requested for the addition of transactions. They expressed that they disliked how much private information needed to be sent to budget and do not view this information as necessary. Chase also mentioned personal information; however, he was frustrated more by the redundancy surrounding this information than privacy. He stated that it is superfluous when banks asks repeatedly for a user’s address on forms when the banks already have this information on file.
%----- END QUESTION 9 -------%

%----- START QUESTION 10 ----%
    \bigskip\noindent\textit{"Is there anything you would like to add?"}

    \bigskip \noindent One user recommended support for various currency types internationally as well as an account option for various cryptocurrencies. This special account type would display information regarding trends in cryptocurrency values and how that value translates into their current holdings similar to stock trading boards. Additionally, the importance of security was mentioned by another user. Extra security is more marketable and will make the service more appealing.
%----- END QUESTION 10 ------%

% EOF